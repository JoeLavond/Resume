\documentclass[10pt]{article}

% display speaking points
\usepackage{etoolbox}
\newtoggle{notes}
\toggletrue{notes}
\togglefalse{notes}

\newtoggle{notnotes}
\toggletrue{notnotes}
\iftoggle{notes}{
    \togglefalse{notnotes}
}

\newcommand\mpfact{.81}
\newcommand\mpfactadj{.82}

\usepackage[T1]{fontenc}

\usepackage{geometry}
\geometry{
    a4paper, 
    margin=.25in, 
    %voffset=.25in
}

\pagenumbering{gobble}
\usepackage{ragged2e}
\justifying

\usepackage{hyperref}
\hypersetup{colorlinks=true}
\urlstyle{same}

\usepackage{multirow}

% Lists
\usepackage{enumitem}

\newenvironment{p_item}{
\begin{itemize}[leftmargin=30pt]
    \vspace{-.2cm}
    \setlength{\itemsep}{1pt}
    \setlength{\parskip}{0pt}
    \setlength{\parsep}{0pt}
}{\end{itemize}}

% Underlines
\usepackage{calc}
\newlength{\remaining}
\newcommand{\titleline}[1]{%
	\setlength{\remaining}{\linewidth-\widthof{\textsc{#1}}}
	\noindent\underline{\textsc{#1}\hspace*{\remaining}}\par
} 

\newcommand{\myheading}[1]{%
    \titleline{\Large{\textbf{#1}}}
    
    \smallskip
}

\newcommand\mysubheading[1]{%
    \begin{minipage}{\mpfact\textwidth}
        \titleline{#1}
    \end{minipage}
   
    \smallskip
}

\newcommand\mydate[1]{%
    \hfill\emph{\small{#1}}
}

\newenvironment{p_item2}{
    \begin{minipage}{\mpfactadj\textwidth}\vspace{.3cm}\begin{p_item}
}{
    \end{p_item}\end{minipage}\medskip
}

%%%%%%%%%%%%%%%%%%%%%%%%%%%%%%%%%%%%%%%%%%%%%%%%%%%%%%%%%%%%%%%%%%%%%%%%   
\begin{document}
\small
	
	% Title 2 
    \iftoggle{notnotes}{%
        \begin{table}[t]
            \centering
            \begin{tabular}{c|l}
                \multirow{4}{*}{\huge{\textbf{JOSEPH LAVOND}}} & Phone: 209-269-7175\\
                & Email: joelavond@gmail.com \\
                & GitHub: \href{https://github.com/JoeLavond}{github.com/JoeLavond} \\
            & LinkedIn: \href{https://linkedin.com/in/josephlavond1997}{linkedin.com/in/josephlavond1997} 
            \end{tabular}
            \vspace{-.3cm}
        \end{table}
    	
    }

    %
    \iftoggle{notes}{%
        \myheading{Summary}
        \textcolor{blue}{
        \vspace{-\baselineskip} \\
            Common Questions:
            \begin{p_item}%
                \item About me: School, advisor, and research, then consulting and internships, finally automotive
                \item Goals: Set industry standards for autonomy, improving access and safety of transportation
                \item Strengths: Background, presentation, work great on a team or as an individual
                \item Weaknesses: Time management and leadership skills
            \end{p_item} 
            My Questions:
            \begin{p_item}
                \item How do you cope with winter? What is your favorite thing to do in February?
                \item What do you see as the next big step for Zero-Zero-Zero?
                \item Current trend for office vs. home work?
            \end{p_item}
        }   
        
        \smallskip
    }
    
    \myheading{Internships}

    %
    \mysubheading{Cisco - San Jose, CA}
    Ph.D. Intern (Remote) \mydate{June 2023 - Aug 2023}
    \begin{p_item2}
        \item Lead research efforts to improve the diversity of samples produced by generative artificial intelligence
        \item Designed architecture changes to generative adversarial networks (GANs) to improve sample diversity 
        \item Identified latent space conditioning as viable strategy for producing higher quality generated outputs from variational auto-encoders (VAEs)
    \end{p_item2}
    
    %
    \mysubheading{Elevance Health, Inc. (Formerly Anthem) - Indianapolis, IN}
    Graduate Info Technology Intern (Remote) \mydate{June 2022 - Aug 2022}
    \begin{p_item2}
        \item Implement custom PyTorch semi-supervised Bayesian anomaly detection approach on Amazon Web Service (AWS) to identify up-coded Evaluation \& Management (E/M) claims 
        \iftoggle{notes}{
        
            \smallskip \textcolor{blue}{
                Situation: Very few known examples of fraud \\
                Task: DS wanted to explore Bayesian methods to tackle use case \\
                Act: Implemented Bayesian model with semi-supervised extension and deep learning solution that approximates complex Bayesian methods \\
                Result: Presentations technical and non-technical with compliments \\
                Reflect: Implement stochastic weight averaging in-house 
            } 
        \smallskip
        }
        
        \item Received Impact award for end-of-summer technical presentation to senior management
        \item Participate in Agile Scrum model development process using Jira of a Fortune-30 company
        \item Create modeling data set pipeline for 60M row database using PySpark SQL in Jupyter on Kubeflow
        \iftoggle{notes}{
        
            \smallskip \textcolor{blue}{
                Situation: Bayesian methods are often computationally costly. However, this model has an exact solution \\
                Task: Compute and store the model solution for fast training and real time prediction \\
                Act: Use PySpark SQL to use Hadoop MapReduce operations to compute solution and store on s3 in AWS \\
                Result: Faster run time than in-house distributed algorithms with minimal performance loss and interpretable solution 
            } 
        \smallskip
        }
        
    \end{p_item2}
    
    %
    \mysubheading{Blue Cross Blue Shield of Arizona - Phoenix, AZ}
    Actuarial Services Intern (Remote) \mydate{Apr 2020 - Aug 2020}
    \begin{p_item2}
        \item Spearheaded the company-wide projection of membership for all lines of business
        \iftoggle{notes}{
        
            \smallskip \textcolor{blue}{
                Situation: Credentialed actuary left already short staffed team early into my internship \\
                Task: Take over their report projecting membership which most finance and actuarial teams build upon \\
                Act: Learned VBA to automate notoriously manual report \\
                Result: Teams now able to complete downstream reports earlier in the month \\
                Reflect: Rewrite report to be readable and understandable
            } 
        \smallskip
        }
        
        \item Learned VBA to automate monthly updates to company forecasts within Microsoft Excel 
        \item Automated data collection using process flows in SAS and SQL queries in Microsoft Access
    \end{p_item2}
    
    
    %
    \iftoggle{notnotes}{
        \myheading{Education}
    	
    	\mysubheading{University of North Carolina - Chapel Hill, NC} 
    	Doctor of Philosophy: \textbf{Statistics and Operations Research}%
    	\mydate{Aug 2020 - May 2025}%
    	\begin{p_item2}%
            \item Funding through NSF-funded grant to add research to theory and application of networks
    	\end{p_item2}
        
    	\mysubheading{California Polytechnic State University - San Luis Obispo, CA}
    	Bachelor of Science: \textbf{Statistics} \mydate{Sep 2016 - Mar 2020}
    	\begin{p_item2}
    		\item 3.99 GPA, Summa Cum Laude, Academic Excellence Award
    		\item Founder and Secretary of the Actuarial Society Club
    		\item Member of Mu Sigma Rho, the US National Statistics Honors Society
    	\end{p_item2}
     }
    	
	
    % 
    \myheading{Research}

    %
    \mysubheading{University of North Carolina - Chapel Hill, NC}
    Research under Dr.\hspace{1pt}Yao Li \mydate{Aug 2021 - Present}
    \begin{p_item2}
        \item Development of methodology to improve the robustness and data privacy of Neural Networks
        \item Proposing Meta-Learning approach to model aggregation for Personalized Federated Learning 
        \iftoggle{notes}{
        
            \smallskip \textcolor{blue}{
                Situation: How to use shared information from all users to create the best personalized models for each \\
                Task: Create a shared model that is a good initialization for user fine tuning \\
                Act: Specific modifications to meta learning algorithms to make viable to federated learning by also having good accuracy for users without data \\
                Result: Implementation that creates improved accuracy on new users with or without data
            } 
        \smallskip
        }
        
        \item Created novel defense against Backdoor Attacks in Federated Learning 
        \iftoggle{notes}{
        
            \smallskip \textcolor{blue}{
                Situation: Users models are aggregated to create a shared model while preserving data privacy, but a single malicious user can send an update that can compromise the shared model \\
                Task: Determine how to identify and exclude malicious updates \\
                Act: With small data set can understand the behavior of non-malicious users and exclude those with different behavior \\
                Result: Can filter out malicious users even when up to 40\% of all suggested updates are harmful \\
                Reflect: Need to extend to adaptive attack, when malicious users know defense
            } 
        \smallskip
        }
        
        \item PyTorch implementation and multi-GPU training of computer vision models on Linux cluster
    \end{p_item2}
    
    % 
    \mysubheading{California Polytechnic State University - San Luis Obispo, CA}
    Statistical Consultant \mydate{Sep 2019 - Aug 2020}
    \begin{p_item2}
        \item Provided SAS mixed modeling analyses to understand the relationships between early life factors and childhood obesity as part of a NIH-funded $\$$2M grant
        \iftoggle{notes}{
        
            \smallskip \textcolor{blue}{
                Situation: Follow mother-infant pairs over time monitoring breast and bottle feeding practices \\
                Task: Statistical inference to determine meaningful relationships in feeding behavior \\
                Act: Create mixed model to remove between subject variation and use backwards selection \\
                Result: Familiarity (frequency bottle feeding and multi-child mothers) have higher intakes \\
                Reflect: Many models can answer the same question, however, some give more desirable interpretations
            } 
        \smallskip
        }
        
        \item Coauthored publication in the Journal of Maternal and Child Nutrition (\href{https://doi.org/10.1111/mcn.13185}{doi:10.1111/mcn.13185})
    \end{p_item2}
    
    %
    Frost Summer Research Program 2019 \mydate{May 2019 - Sep 2019}
    \begin{p_item2}
        \item Manipulate the Surveillance, Epidemiology, and End Results (SEER) database, containing over 20 million cancer cases within the United States, using SAS
        \item Analyze SEER data with KM\iftoggle{notes}{ \textcolor{blue}{(Empirical survival curve)}}{} and COX\iftoggle{notes}{ \textcolor{blue}{(Proportional hazards = instantaneous failure rate)}}{} survival models using R
        \iftoggle{notes}{
        
            \smallskip \textcolor{blue}{
                Situation: Large database on many historical cancers cases in US \\
                Task: Present sensitive information in an informative manner \\
                Act: Coordinate use of multiple statistical tools to process and model large data \\
                Result: Interactive interface to visualize prognosis and impact of various treatment options \\
                Reflect: More complex models needed to give accurate estimates
            } 
        \smallskip
        }
        
        \item Construct an interactive R Shiny app for cancer patient use, taking demographic and diagnosis information as inputs, and providing a prognosis
    \end{p_item2}
    
    
    
    %
    \myheading{Skills}
    
    \mysubheading{Coursework}
    \begin{p_item2}
        \item Machine Learning: Gradient Boosting\iftoggle{notes}{ \textcolor{blue}{(Penalize mistakes heavier for later models)}}, Bagging\iftoggle{notes}{ \textcolor{blue}{(Bootstrap aggregation)}}, Bootstrapping, Regularization\iftoggle{notes}{ \textcolor{blue}{(Penalize against large or unwanted solutions)}}, Dimension Reduction 
        \item Deep Learning: Semi-Supervised\iftoggle{notes}{ \textcolor{blue}{(Partially labeled)}}, Self-Supervised\iftoggle{notes}{ \textcolor{blue}{(Learn pseudo-labels)}}, and Transfer Learning for NLP and Computer Vision
    \end{p_item2}
    
    \mysubheading{Programming Languages}
    \begin{p_item2}
        \item SQL (Snowflake, HUE, Microsoft Access) 
        \item Python (PyTorch, SciKit-Learn, PySpark, Pandas, NumPy, Matplotlib) 
        \item R (caret, tidyverse, ggplot2, dplyr, purr, stringr, shiny)
        \item Other: SAS, Microsoft Excel
    \end{p_item2}
    
    \textsc{Misc.} Git (GitHub, Bitbucket), Linux (Cluster Computing, GPU Acceleration), Vim, Project Jupyter, Kubeflow
    
    
    
\end{document}
    	